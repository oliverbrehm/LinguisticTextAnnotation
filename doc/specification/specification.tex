\documentclass[twoside,12pt,a4paper]{scrreprt}
\usepackage[T1]{fontenc}
\usepackage[utf8]{inputenc}
\usepackage[ngerman]{babel}
\usepackage{babelbib}
\usepackage{parskip}
\usepackage{microtype}
\usepackage{graphicx} % Zum Einbinden von Grafiken
\usepackage[dvipsnames]{xcolor}
\usepackage[colorlinks=true,linkcolor=Black,citecolor=MidnightBlue,urlcolor=MidnightBlue]{hyperref}
\usepackage[all]{hypcap}
\usepackage{pgfplots} \pgfplotsset{compat=1.9}
\usepackage{helvet} % Schönere SansSerif-Schrift
\usepackage{times}  % Schönere Serif-Schrift

\usepackage{blindtext} % sollte am Ende nicht mehr benötigt werden ;)

\pagestyle{headings}

\graphicspath{ {figures/} } % Pfad-Prefix für einzubindende Grafiken. Es sind auch mehrere Pfade möglich, diese müssen jeweils in eigenen {Klammern} stehen.

\setkomafont{disposition}{\normalcolor\bfseries} % überall Serifen verwenden
% oder
%\renewcommand{\familydefault}{\sfdefault} % überall Sans-Serif verwenden

% PDF-Optionen (werden in den Dateieigenschaften angezeigt)
\hypersetup{
pdftitle={Entwicklung einer Webanwendung zur Annotation spezifischer linguistischer Merkmale in Fließtexten},
pdfauthor={Oliver Brehm},
pdfsubject={Masterarbeit Infromatik},
pdfpagelayout=TwoColumnRight
}

%%% Eigene Makros
\newcommand{\qq}[1]{\glqq #1\grqq} % \qq{Text in Anführungszeichen}

\begin{document}

%%% Titelseite
\begin{titlepage}
\begin{center}
\LARGE Eberhard Karls Universität Tübingen\\
\large Mathematisch-Naturwissenschaftliche Fakultät \\
Wilhelm-Schickard-Institut für Informatik\\
[3cm]
\huge LinguisticTextAnnotation\\
[2cm]
\Large\textbf{Specification}\\
[1.5cm]
\large Oliver Brehm\\
[0.5cm]
31.01.2017\\
\vfill

\end{center}
\end{titlepage}

\tableofcontents\label{toc}
\cleardoublepage

\section{Requirements}

\paragraph{Text Analyse} 
Der User möchte einen Fliestext eingeben und von der Anwendung annotieren lassen. Nach der Eingabe soll das Ergebnis annotiert in der App dargestellt werden.

\paragraph{Anpassung der annotierten Darstellung}
Der User möchte die Parameter des annotierten Texts ändern. Angepasst werden können sollen Texteigenschaften (Font, Zeilenabstand), Darstellung der Annotation (Farben, Fonts für betonte und unbetonte Silbe), Zusätzliche Visualisierung für die Wortsegmentierung (Silbenbögen, Klammern, Abstände)

\paragraph{Export der annotierten Darstellung}
Der User hat die Möglichkeit verschiedene Formate des annotierten Texts zu exportieren, z.B. Druck, HTML oder Word.

\paragraph{Verwaltung von User Accounts}
Dem User soll die Möglichkeit gegeben werden, einen User Account zu erstellen, um persönlch verwendete Daten (z.B. Texte, Wortsegmentierungen) speichern zu können. Dazu müssen Funktionen und Interfaces für Regestrieren eines Nutzeraccounts, Login, Logout, Bearbeiten der Nutzderinformationen und Löschen des Accounts bereitgestellt werden.

\paragraph{Behandlung unbekannter Wörter}
Dem User soll durch Klicken auf ein Wort oder einen Button "Unbekannte Wörter hinzufügen" die Möglichkeit gegeben werden, nacheinander die Segmentierung von Wörtern, die durch das System nicht eindeutig bestimmt wurden konnten, selbst festlegen zu können.

\paragraph{Bestimmung der Segmentierung unbekannter Wörter}
Für ein unbekanntes Wort soll in einem neuen View die segmentierung ausgewählt werden können. Dafür werden folgende Möglichkeiten gegeben:
\begin{itemize}
	\item Segmentierungssystem aus der Bachelorarbeit von Leona Göbbels (CITE)
	\item Input aus G2P Systemen wie MARY
	\item Manuelle Segmentierung mit geeignetem User Interface
\end{itemize}

\paragraph{Speicherung von Nutzer Segmentierungen}
Vom Nutzer hinzugefügte Segmentierungen sollen (lokal für diesen Nutzer) gespeichert werden können und beim nächsten Vorkommen in einem Text automatisch verwendet werden.

\paragraph{Speicherung von Annotationskonfigurationen}
Die Einstellungen, die ein Nutzer an einenm annotierten Text vorgenommen hat, können als Vorlage für andere Texte gespeichert werden.

\paragraph{Auswahl einer Annotationskonfiguration für einen Text}
In den Annotationseinstellungen eines Textes kann eine zuvor gespeicherte Konfiguration verwendet werden.

\paragraph{Speicherung von Nutzertexten}
Analysierte Texte können vom Nutzer zusammen mit der verwendeten Konfiguration gespeichert werden. Den Texten können Metadaten zugeordnet werden, z.B. Thema, Niveau, Zielgruppe.

\paragraph{Auflistung von Nuetzertexten}
Im Benutzerbereich werden die Texte, die der Nutzer hinzugefügt hat, geeignet strukturiert, dargestellt.

\paragraph{expert user functions}
TODO

\cleardoublepage

\section{Software Komponenten}

\subsection{Überblick}

Grobstruktur der Komponenten:

\begin{itemize}
	\item RESTful python backend
		\subitem REST API
		\subitem Dictionary Service -> Word DB
		\subitem User Service -> User DB
	\item Web frontend: (AngularDart application)
		\subitem TextAnnotation Component
		\subitem User Component
\end{itemize}

\subsection{Komponenten}
Aufbau der Komponenten im Detail:

TODO
\cleardoublepage

\section{Datenmodell}

Aus den Anforderungen lässt sich folgendes Datenmodell ableiten:

TODO
\cleardoublepage

\section{User Interface}

TODO user interface designs for each requirement
\cleardoublepage

\end{document}