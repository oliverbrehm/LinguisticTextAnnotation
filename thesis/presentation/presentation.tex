\documentclass{beamer}

\definecolor{beaver}{HTML}{8F0000}

\mode<presentation>
{
	\usetheme{Berlin}      % or try Darmstadt, Madrid, Warsaw, ...
	\usecolortheme{beaver} % or try albatross, beaver, crane, ...
	\usefonttheme{default}  % or try serif, structurebold, ...
	\setbeamertemplate{navigation symbols}{}
	\useoutertheme[subsection=false]{smoothbars}
	\setbeamertemplate{itemize item}{\color{beaver}$\blacksquare$}
} 

\usepackage[ngerman]{babel}
\usepackage[utf8x]{inputenc}
\usepackage[T1]{fontenc}
\usepackage{xcolor}

\usepackage{amsmath}
\usepackage{graphicx}


\title{Entwicklung einer Webanwendung zur Annotation spezifischer linguistischer Merkmale in Fließtexten}

\subtitle{Masterarbeit Medieninformatik}
\author{Oliver Brehm}
\institute[Eberhard Karls Universität Tübingen] 
{
	Eberhard Karls Universität Tübingen\\
	Mathematisch-Naturwissenschaftliche Fakultät \\
	Wilhelm-Schickard-Institut für Informatik
}
\date{14. Februar 2018}

\begin{document}
	
\begin{frame}[plain]
	\titlepage
\end{frame}

\begin{frame}{Motivation}
\begin{itemize}
	\item \textbf{Lese- Rechtschreibschwäche}: Weit verbreitete Entwicklungsstörung
	\item Übungstexte mit Hervorhebung von Silben in verschiedenen Farben (Unterstützung bei der Wortdekodierung, Sprachrhythmus)
	\item \textbf{Zielgruppe}: LerntherapeutInnen, Linguisten, Eltern betroffener Kinder
	\item Webanwendung mit Fokus auf User Experience
\end{itemize}
\end{frame}

\begin{frame}{Eine Webanwendung zur Textannotation}
\centering
\includegraphics[height=0.8\textheight]{../figures/frontend/textanalyse}
\end{frame}

\section{Grundlagen}
\begin{frame}
	\centering
	\huge{Grundlagen und Forschung}
\end{frame}

\begin{frame}{Vergleichbare Projekte}
\begin{itemize}
	\item ABC der Tiere Silbengenerator
\end{itemize}

\centering
\includegraphics[height=0.6\textheight]{../figures/ABCsilbengenerator}

\begin{itemize}
	\item celeco Druckstation
	\item an Verlage und Produkte gebunden
\end{itemize}
\end{frame}

\begin{frame}{Linguistische Analyse}
\begin{itemize}
	\item beliebiger Eingabetext
	\item Parser/Tokenizer (Spacy mit Python)
	\item Tagging (Part-of-Speech, Lemma)
\end{itemize}
\vfill
\includegraphics[height=0.5\textheight]{../figures/json}
\end{frame}

\begin{frame}{Wortdatenbank}
\textbf{Silbentrennung und Wortbetonung gesucht}
\begin{itemize}
	\item Silbentrennung aus Hyphenator (Pyphen)
	\item \textbf{Wortbetonung}: Lexikon CELEX2 als Goldstandard
	\item Bestimmung der Betonung aus Phonologischer Darstellung (z.b. \textit{antizipiert | \&n-ti-=i-'pirt})
	\item Nicht alle Wörter (mit Flexionsformen) enthalten
	\item Generieren von Vorschlägen bei unbekannten Wörtern
	\item Text-to-Speech Systeme (z.B. MARY)
\end{itemize}
\end{frame}

\section{Entwicklung der Anwendung}
\begin{frame}
\centering
\huge{Entwicklung der Anwendung}
\end{frame}

\begin{frame}{Anforderungsanalyse}
Erarbeitung der Anforderungen in Expertengesprächen:
\begin{itemize}
	\item Verwaltung der Wortdatenbank
	\item Nutzerverwaltung
	\item Textanalyse
	\item Nutzer: Registrierung, Anmeldung, Speicherung von Texten und Einstellungen
	\item Verifizierung neuer Datenbankeinträge durch NutzerInnen
\end{itemize}
\end{frame}

\begin{frame}{Aufbau der Anwendung}
	\centering
	\includegraphics[width=0.85\textwidth]{../figures/frontendbackend}\\
	\begin{itemize}
		\item RESTful API mit Python und Flask
		\item Wort- und User-Datenbanken (SQLite)
		\item Alternativen recherchiert (NoSQL, Backend-as-a-Service)
	\end{itemize}
\end{frame}

\begin{frame}{Backend}
	\centering
	\includegraphics[width=0.95\textwidth]{../figures/backend}
\end{frame}

\begin{frame}{Frontend}
\begin{itemize}
	\item Single-Page-Application
	\item AngularDart Framework
	\item Datenmodell in Dart Klassen
	\item HTML/CSS Templates mit Data Binding
\end{itemize}
\centering
\includegraphics[width=0.95\textwidth]{../figures/frontend/uml-annotationtext}
\end{frame}

\section{Evaluation}
\begin{frame}
\centering
\huge{Evaluation}
\end{frame}

\begin{frame}{Ergebnisse der Textannotation}
\begin{columns}[t]
	\column{.5\textwidth}
	\centering
	\includegraphics[width=5.5cm]{../figures/evaluation/pre-annotation1}\\
	[0.5cm]
	Einfache, abwechselnde Darstellung
	\column{.5\textwidth}
	\centering
	\includegraphics[width=5.5cm]{../figures/evaluation/pre-annotation3}\\
	[0.5cm]
	Andere Farben, manuelle Aktivierung der Betonung
\end{columns}
\end{frame}

\begin{frame}{Nutzertest}
\begin{itemize}
	\item Interview mit \textit{Thinking Aloud}
	\item Zusätzlich \textit{After Scenqrio Questionnaire} für quantitative Daten
	\item Pilottest
	\item Sieben ProbandInnen aus zwei Gruppen
\end{itemize}
\end{frame}

\begin{frame}{Ergebnisse der Szenarien}
\begin{columns}[t]
	\column{.5\textwidth}
	\centering
	\includegraphics[width=5.5cm]{../figures/evaluation/scenario1}\\
	1: Nutzerkonto\\
	[1.5cm]
	\includegraphics[width=5.5cm]{../figures/evaluation/scenario2}\\
	2: Textanalyse
	\column{.5\textwidth}
	\centering
	\includegraphics[width=5.5cm]{../figures/evaluation/scenario3}\\
	3: Annotationsvorlagen\\
	[1.5cm]
	\includegraphics[width=5.5cm]{../figures/evaluation/scenario4}\\
	4: Texte wiederverwenden
\end{columns}
\end{frame}

\begin{frame}{Fazit}
\begin{itemize}
	\item Automatische Erstellung von Übungstexten, Arbeitserleichterung
	\item Erweiterbare Wortdatenbank durch Crowdsourcing
	\item Intuitive Bedienbarkeit
	\item Erweiterung möglich (Modul für SchülerInnen, mobile App...)
\end{itemize}
\end{frame}

\begin{frame}[plain]
\centering
\huge{Danke für die Aufmerksamkeit!}
\\
[1cm]
\huge{\color{beaver}{$\rightarrow$ Fragen und Demo}}
\end{frame}

\end{document}