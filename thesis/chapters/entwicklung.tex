% !TEX root = ../ausarbeitung.tex

\chapter{Entwicklung der Anwendung}

Überblick, welche Schritte wurden wann erledigt?\\
Aufbau der Datenbank\\
REST API, verschiedene Services\\
Frontend, verschiedene Komponenten (Hauptnavigationskomponenten, nicht angular komponenten)\\

\section{Basisdatenbank}

CELEX\\
was? beschreibung\\
alternativen?\\
warum?

\subsection{Impementierung celex2db}

parsing\\
modell, zusammenführung verschiedener cd dateien\\
sqlite3, word Tabelle\\

\subsection{Ergebnisse}

anzahl wörter\\
fehler\\
zeit, aber irrelevant weil nur einmal erledigen


\section{Backend}

flask, implementierung der Routen s. spezifikation\\
überblick services, wer macht was, abhängigkeiten (Diagramm)\\

\subsection{Routen}
angebotene funktionen, routen mit parameter namen

\subsection{Dictionary Service}

was macht dieser?\\
sqlite Tabellen\\
Klassendiagramme\\

\subsection{User Service}
was macht dieser?\\
sqlite Tabellen\\
Klassendiagramme\\

\subsection{Verification Service}
was macht dieser?\\
sqlite Tabellen\\
Klassendiagramme\\


\section{Frontend}

angular dart app\\
komponenten s. software stack erklärung\\

app\_component, providers, directives\\

\subsection{Model}
word, syllable...

\subsection{Begrüßungsseite}

home\_component\\
angular router links

\subsection{Text Analyse}

ui und service

\subsubsection{TextAnalysisService}

funktionen, model abhängigkeiten

\subsubsection{Texteingabe}

texteingabe, material-input\\

\subsubsection{Textvorschau}

iterieren über wörter\\
span je nach word state\\
notfound wörter mit link zu word review\\
linebreaks (sind wörter, spezieller state)\\

editierbare wörter, css aus komponente\\
popup:\\
\begin{itemize}
	\item annotieren oder nicht
	\item manuelles Betonungsmuster
	\item manuelle Silbentrennung
	\item gleiche wörter übernehmen
	\item Wortart
	\item Lemma
\end{itemize}

span für silben\\
style aus komponente\\
separator zeichen\\

drucken, css vorlage, @media\\
text kopieren, js/dart magic\\

\subsubsection{Textoptionen}

text neu analysieren\\
text speichern, mit titel\\
neuen text eingeben

\subsection{Annotationseinstellungen}

verschiedene einstellungen\\
in server für user gespeichert\\

\subsubsection{TextConfigurationService}

TODO refactor, diesen service neben user service anlegen

\subsubsection{User Interfache für Einstellungen}
betonte silbe\\
fett, farbe\\

color picker\\

unbetonte silben\\
farbe\\
2. farbe, benutzen\\

worthintergrund, warum, in welcher einstellung sinnvoll?

größen und abstände\\
\begin{itemize}
	\item Schriftgröße
	\item Silbenabstand
	\item Wortabstand
	\item Zeilenabstand
	\item Zeichenabstand
\end{itemize}

custom slider\\

Silbentrennzeichen\\
warum, benutzen, eingabe\\

Wortart dropdown\\
funktionen separat wür die verschiedenen Wortarten\\
dropdown\\
funktion iteriert über alle wörter im text und wendet eigenschaft auf jedes passende Wort an\\
funktionen:\\
\begin{itemize}
	\item Annotieren
	\item Unbetont
	\item Ignorieren
\end{itemize}

\subsubsection{Annotationsvorlagen}

speichert alles\\
name der vorlage\\
liste, aktivieren, löschen\\
neue vorlage anlegen oder momentane speichern\\

\subsection{Text Review}

TBD

\subsection{Nutzerkonto}

TBD

\subsection{Wort Verifizierung}

TBD