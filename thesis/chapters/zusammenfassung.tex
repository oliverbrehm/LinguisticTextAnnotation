% !TEX root = ../ausarbeitung.tex

\begin{abstract}
\section*{Zusammenfassung}

Viele Menschen haben Schwierigkeiten, flüssig zu Lesen und zu Schreiben. Man spricht von einer Lese- Rechtschreibschwäche (LRS), wenn keine erkennbaren äußeren Umstände (wie soziales Umfeld, oder längerer Schulausfall z.B. durch Krankheit) die Entwicklung der Lese- und Schreibkompetenz beeinträchtigen. Es wurde gezeigt, dass gezielte Übungen wie das Erfassen von Silben und Sprachrhythmus dazu führen können, diese Fähigkeiten zu verbessern. In der Therapie werden häufig Übungstexte verwendet, in denen Silben abwechselnd in verschiedenen Schriftfarben dargestellt werden.

In dieser Arbeit wurde eine Web-Applikation entwickelt, mit der es ermöglicht werden sollte, solche farblich markierten Texte einfach und automatisch zu erstellen. Für die Verdeutlichung des Sprachrhythmus sollte auch die betone Silbe im Wort speziell markiert werden können. Es wurde zunächst untersucht, wie aus einem beliebigen Text automatisch Silbentrennung und Wortbetonung bestimmt werden können. Dazu wurde eine, auf einem Lexikon basierende Datenbank aufgebaut, die für jedes Wort diese Merkmale enthält. Die Datenbank ist durch NutzerInnen der Anwendung erweiterbar, hinzugefügte Einträge müssen jedoch von ExpertInnen oder anderen NutzerInnen verifiziert werden, damit sie global verfügbar sind. Für die Applikation wurden zunächst die Anforderungen analysiert und anschließend zwei Hauptkomponenten entwickelt: Auf der einen Seite das Backend, welches Anfragen der Web-Applikation, wie z.B. den zu analysierenden Text entgegen nimmt und beantwortet, sowie die erstellte Wortdatenbank und eine weitere Datenbank, die nutzerspezifische Einstellungen speichert, verwaltet. Auf der anderen Seite steht die Entwicklung des Frontends, der eigentlichen Web-Applikation, welche mit NutzerInnen interagiert, Anfragen an das Backend schickt und die dadurch erhaltenen Informationen entsprechend darstellt.

Die abschließende Evaluation zeigt, dass die Applikation Übungstexte erfolgreich erstellen kann. Mit einem Nutzertest wurde festgestellt, dass die meisten Anwendungsfälle in der Web-Oberfläche intuitiv und zeiteffizient durchgeführt werden konnten. Die erkannten Probleme können durch eine Überarbeitung der Nutzeroberfläche mit wenig Aufwand behoben werden. Weitere Anmerkungen der ProbandInnen generierten interessante, weiterführende Ideen, die auf diese Arbeit aufbauend gut realisierbar sind.

\end{abstract}