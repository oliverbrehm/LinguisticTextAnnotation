% !TEX root = ../ausarbeitung.tex

\chapter{Stand der Forschung}

Die Motivation der Arbeit beruht auf Erkenntnissen zu Behandlungsformen für Lese- Rechtschreibschwäche (LRS). Es wird untersucht wie gut der Prozess des Erzeugens von Übungstexten automatisiert werden kann. Hierfür war das Ziel bei der Entwicklung der App eine möglichst gute User Experience zu bieten. Die Zielgruppe, vorwiegend bestehend aus Lerntherapeuten, Nachhilfelehrern und Eltern betroffener Kinder soll das Programm möglichst intuitiv bedienen können.\\
Im Folgenden werden daher erst die Grundlagen zu LRS erklärt und Ansätze zur Digitalisierung von Übungstexten, z.B. auch ähnliche, bereits vorhandene Software, beschrieben. Danach werden die computerlinguistischen technischen Grundlagen erklärt, die in der Entwicklung der Applikation benötigt werden.

\section{Lese- Rechtschreibschwäche}

Ein Ausgangspunkt zur Feststellung von LRS bietet die Lesekompetenz. Eine Definition aus der PISA Studie \tocite{PISA, SK s51} untergliedert die Lesekompetenz in die vier Teilbereiche \textit{Kognitive Grundfähigkeit}, \textit{Dekodierfähigkeit}, \textit{Lernstrategiewissen} und \textit{Leseinteresse}.\\
Der Teilbereich, zu dessen Verbesserung diese Arbeit einen Beitrag leisten kann ist die Dekodierfähigkeit. Diese Stellt die Kompetenz dar, die Bedeutung von Wörtern, Sätzen und Texten zu erfassen und zu verstehen. Die Unterteilung in dei Elemente Wort, Satz und Text stellt eine Dekodierungsfähigkeit auf verschiedenen Ebenen dar, die sich gegenseitig bedingen. Das Wort ist hier die Basisebene, so kann ein Text nur verstanden werden, wenn die Bedeutung der einzelnen Sätze erfasst wurde und diese wiederum wird nur erkannt, wenn eine Ausreichende Dekodierfähigkeit für Wörter vorhanden ist.\\
Somit ist eine Vorraussetzung für die Verbesserung des Textverständnis, dass das Erkennen des Basiselements Wort beherrscht wird.

\subsection{Dekodierung des Wortbildes}

Es ist bekannt, dass Kinder der Entwicklung des Lesens und des Schreibens verschiedene Phasen durchlaufen. \tocite{quelle hier sehr weit weg (steinbrink), vielleicht frueher bringen, weitere quelle} So gibt in jedem Fall eine alphabetische und eine orthographische Phase. Beim Lesen wird in der alphabetischen Phase ein Wort Buchstabe für Buchstabe dekodiert, \textit{Grapheme} (die kleinste Einheit in der Schriftsprache, in deutsch Buchstaben) werden einzeln in \textit{Phoneme} (Laute, die kleinste Einheit der gesprochenen Sprache) umgewandelt, was aber bei nicht lautgetreuen Wörtern (Wörtern, in denen nicht alle Grapheme korrekt in die zugehörigen Phoneme übersetzen lassen) nicht gelingt.\\
Daher wird später in der orthographischen Phase eine Strategie benutzt, die sich an größeren Bestandteilen orientiert. Wörter oder Wortteile werden hier aus dem Langzeitgedächtnis abgerufen, was bei richtig gelernten Silben und Wörtern zu korrekter Aussprache führt. \cite{Steinbrink2014}\\
Einige schulische Ansätze fördern das orthographische Lernen mit einer gezielten Hervorhebung von Silben. \tocite{ABC der Tiere, Silbenfibeln}. \todo{mehr dekodierung stuff} Viele Fördermaterialien sind erhältlich, die Silben beispielsweise farblich hervorheben und damit guten Lernerfolg erzielen. \tocite{ABC}
Weiterführende Arbeiten legen aber auch nahe, dass der Sprachrhythmus eine zentrale Rollen sowohl beim Lesen als auch beim Schreiben spielt. So hat ein Wort in der deutschen Sprache eine oder mehrere Betonungen. Betonte Silben werden, im Gegensatz zu den unbetonten Silben lauter und länger gesprochen. Es wurde gezeigt, dass gezieltes Training des Sprachrhythmus in Verbindung mit der orthographischen Repräsentation eines Wortes die Dekodierfähigkeit steigern kann. \cite{Brandelik2014}
\todo{more betonung stuff}

warum ist zeicheinabstand wichtig? mehr zu font, textgröße, wortabstand... -> Katharina?

\subsection{Therapiemethoden}

Wie kann LRS sonst noch behandelt werden?\\
Methoden zum Lesetraining\\
Farbliche Hervorhebung, Silbenmethode ABC der Tiere, andere quellen
Literatur zu übungen\\

\subsection{Digitale Ansätze}
Vorhandene Methoden\\
ABC der Tiere software, CELESCO ( https://www.celeco.de/?page=LRS\_Texte\_mit\_Silbenboegen\_drucken)
Digitale Tools\\

vorhandene tools vorstellen, Literatur zu Tools, zeigen, wie diese Ideen weitergeführt werden können, übergang zur notwendigkeit des eigenen tools\\

\section{Computerlinguistische Grundlagen}

Um das Ziel einer automatischen Analyse und Annotation beliebiger Texte zu realisieren, werden computerlinguistisch vor Allem die folgenden zwei Schritte untersucht:
\begin{enumerate}
	\item Syntaktische Analyse (Zerlegung des Texts in Sätze und Wörter)
	\item Nachschlagen der Betonungsmuster in einer Datenbank
\end{enumerate}

\subsection{Syntaktische Analyse}
was? wie funktioniet das?
grundbegriffe aus carstensen\cite{Carstensen2009}
nlp parser python openNLP, spacy\\
was liefern diese? zusätzlich lemma, pos\\

\subsection{Datenbanken zur Wortbetoung}
Silbentrennung\\
Wortbetonung\\
Part of speech, was, warum? doppeldeutigkeiten?\\
Lemma, was ist das\\
CELEX, Alternativen\\
Graphem2Phonem beschreiben, warum?\\
Tools, MARY, BAS...

Erweiterung durch Crowdsourcing, viele Mithelfer durch Plattformen wie Amazon Mechanical Turc\cite{Snow2008} oder CrowdFlower \todo{Meurers} \cite{Meurers2015}, \todo{translation}\cite{Zaidan2011}.