% !TEX root = ../ausarbeitung.tex

\chapter{Einleitung}

was: LRS/Legasthenie, Methoden zur Therapie, meist manuell erstellte übungen, Literatur zu übungen\\
kurz vorhandene tools vorstellen, zeigen, diese sind nicht auf das spezifische problem zugeschnitten, Literatur zu Tools\\

warum: einsparung von arbeit der lerntherapheuten, automatisierun, digitalisierung\\
therapie von LRS wichtig, einfacheres erstellen von übungen, mehr übungen, mehr lernzeit, weniger betreung, insgesamt mehr lernerfolg (keine belege, prognose, oder mit anderen lerntools belegen)
wenig gut benutzbare tools gerade im bildungsbereich (belegen, negativ und positiv beispiele)\\
vereinheitlichung von übungen, wie werden diese sonst von verschiedenen Therapheuten erstellet? welche vorgehensweisen gibt es?\\
schwierig weil: \\
silbenbetonung nicht als service/lexikon/datenbank vorhanden, eigene datenbank schwierig, g2p tools: woher beziehen diese ihre daten?\\
erarbeitung einer spezifikation mit fachkundigen experten? \\
wie wurden features identifiziert: brainsorming, interview, fragebogen


wissenschaftliche Beiträge: \\
untersuchung von wortbetonung, wie kann eine datenbank erstellt und erweitert werden\\
einfache benutzbarkeit einer anwendung für bildung, webapplikation zur plattformunabhängigen benutzung\\
ziel möglichst gute usability, durch evaluation mit therapheuten und fachfremden bestätigt

ausblick\\
einführung in LRS, methoden zum lesetraining\\
spezifikation und wahl des software stacks, frontend, backend, technologien\\
entwicklung, beschreibung der komponenten und module
evaluation (was ist dabei klar geworden?) und ausblick (nur 1/2 punkte hier nennen)