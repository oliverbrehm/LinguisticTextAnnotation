% !TEX root = ../ausarbeitung.tex

\chapter{Einleitung}

In der vorliegenden Arbeit wurde eine Applikation entwickelt, die Fließtexte durch farbliche Annotationen so aufbereiten kann, dass diese als Übungstexte in der Therapie von Kindern mit Lese-Rechtschreibschwäche (LRS) eingesetzt werden können. Solche Übungstexte werden häufig von LerntherapeutInnen manuell, z.B. mit einem Textverarbeitungsprogramm mit viel Aufwand erstellt. Die Zielgruppe für die Benutzung der Applikation sind also LerntherapeutInnen, Lehrkräfte oder auch die Eltern betroffener Kinder, deren Arbeit durch die Automatisierung des Prozesses der Erstellung von Übungstexten erleichtert werden kann.\\

LRS tritt bei ca. 700000 Kindern und Jugendlichen unter 18 Jahren auf \cite{Schulte-Koerne2014} \todo{andere zahlen, andere quelle}. Bei der Therapie gibt es verschiedene Maßnahmen um die Lese- und Schreibkompetenz von Betroffenen zu fördern. Das Lesen speziell bearbeiteter Texte zielt z.B. darauf ab, die Dekodierfähigkeit zu verbessern, was neben anderen Faktoren einen Teil der Ursachen für LRS abdeckt\cite{Schulte-Koerne2014}. Es wurde gezeigt, dass beim fortgeschrittenen Lesen und Schreiben nicht Buchstaben und Laute die erfassten Elemente sind, sondern größere orthographische Einheiten, wie Silben oder ganze Wörter.\cite{Steinbrink2014} Eine Zentrale Rolle spielt dabei das Erkennen von Silben. Die Silbenmethode und Silbenfibeln sind beim Schrifterwerb im schulischen Umfeld im Einsatz \todo{seit wann?} und können Erfolge aufweisen\tocite{Silbenmethode, einsatz, Bredel, Tophinke}. Eine weitere Erkenntnis hat gezeigt, dass die Wahrnehmung von Sprachrhythmus ebenfalls eine Auswirkung auf die Lese- und Schreibkompetenz hat\cite{Brandelik2014}. Es wird vermutet, dass ein gezieltes Training der Bewusstheit von Silbenbetonung zu höherem Lernerfolg führen kann\cite{Holz2017} (s. Abschnitt \ref{sec:dekodierung}). Im Folgenden Kapitel werden auch einige Alternativen vorgestellt, die das Lesen Lernen mit der Silbenmethode unterstützen.\\

Basierend auf diesen Überlegungen wurde die Applikation mit der Motivation entwickelt, das Erstellen von Übungstexten zu vereinfachen und zu automatisieren. In den Übungstexten werden Wörter in Silben unterteilt, welche sich mit verschiedenen Farben hervorheben lassen. Im Gegensatz zu ähnlichen Ansätzen (sowohl Lehrmaterial in Papierform als auch digitale Ansätze) bietet die Applikation zusätzlich die Möglichkeit, die betonte Silbe im Wort besonders (z.B. in einer anderen Farbe) darzustellen. Weitere Merkmale, wie der Abstand zwischen Silben und Wörtern, sowie ein Silbentrennzeichen, lassen sich frei anpassen. Aktuell werden solche Texte oft manuell, z.B. mit einem Textverarbeitungsprogramm erstellt. Die Anwendung kann TherapeutInnen, Lehrkräften oder Eltern helfen, spezifisch zugeschnittene Übungen mit wenig Aufwand selbst zu erstellen.\\

Es können viele Gründe genannt werden, weshalb wir uns von der Entwicklung einer solchen Anwendung Erfolg versprachen: Ein Grund ist die effizientere Nutzung von Ressourcen. Automatisierung von Aufgaben bringt immer einen Zeitgewinn für die Personen, die diese sonst erledigten mit sich. Zeit, welche die LerntherapeutInnen beim Erstellen von Übungsmaterial einsparen kann z.B. für eine Bessere Vorbereitung der Therapie oder zu mehr Betreuungszeit von SchülerInnen führen. Außerdem ist die Gestaltung von Übungsmaterial sehr flexibel, da in der Software viele Einstellungen für die Textdarstellung vorgenommen werden können. So können die LerntherapeutInnen einfach verschiedene Möglichkeiten beim Erstellen von Übungstexten ausprobieren und bei der Verwendung mit den SchülerInnen deren Präferenzen erkennen. Dadurch lassen sich eventuell direkt Erkenntnisse gewinnen, welche Art von Textannotation am besten funktioniert.\\
Bei der Recherche konnte nur wenig vergleichbare Software gefunden werden. Tools, die ebenfalls die Aufgabe der Annotation von Silben in Texten erledigten sind zum Teil veraltet oder bieten wenig Möglichkeiten zur Anpassung der Annotation (s. Abschnitt \ref{sec:lrs-digital}).

Die Entwicklung der Anwendung stellte verschiedene Herausforderungen in diversen Bereichen dar. Zuerst musste untersucht werden, wie in Wörtern Silbentrennung und Wortbetonung automatisch bestimmt werden kann. Hierfür wurde eine eigene Datenbank aufgebaut, die diese und andere Merkmale speichert. Eine besondere Schwierigkeit stellte die Komplexität und der Umfang der deutschen Sprache dar. Als Grundlage für die Datenbank diente das Lexikon CELEX2 \tocite{CELEX2}, aus dem rund 360 000 Einträge, welche sowohl aus Grund- als auch Flexionsformen von Wörtern bestehen, extrahiert werden konnten. Damit ist es aber nicht möglich den deutschen Wortschatz lückenlos mit allen Flexionsformen abzubilden. Es wurde deshalb nach einer Möglichkeit gesucht, eine benutzerfreundliche Schnittstelle zu bieten, mit der korrekte Einträge von NutzerInnen und Experten manuell, mit wenig Aufwand der Datenbank hinzugefügt werden können. Es wurde untersucht, welche Ansätze es gibt, mit \textit{Croudsourcing} die Erweiterbarkeit der Datenbank auf die NutzerInnen aufzuteilen (s.Abschnitt \ref{sec:forschung-database}). So könnten beispielsweise auch Eigennamen, die in häufig als Übungstexte verwendeten Geschichten häufig auftauchen, Teil des Wortschatzes werden.\\

Vor der Entwicklungsphase mussten der Rahmen und Funktionsumfang der Software definiert und die einzelnen Komponenten spezifiziert werden. Dafür wurde zunächst eine Anforderungsanalyse durchgeführt. Hierbei wurden Ideen aus verschiedenen Quellen gesammelt, es wurden Brainstormings in der eigenen Forschungsgruppe durchgeführt, sowie Gespräche mit Experten aus der Computerlinguistik und der Lerntherapie geführt. Es wurden alle wichtigen Aspekte der Applikation wie Design, Arten der Textannotation, Funktionen der Wortdatebank und des Nutzerkontos etc. diskutiert. So entstanden viele Ideen und Szenarien woraus dann eine Spezifikation erarbeitet werden konnte. In einer späteren Phase der Entwicklung mussten hier noch nachgebessert werden. Es entstanden parallel noch weiter Ideen für Features, die am Anfang nicht vorgesehen waren. Daher wurden die ins Projekt involvierten Personen zusätzlich mit einem Fragebogen gebeten, die Wichtigkeit weiterer Features zu bewerten und anzugeben mit welcher Priorität diese noch entwickelt werden sollten.\\

Einen weiteren wichtigen Punkt stellte die User Experience dar. Die Zielgruppe der Nutzer, welche die Applikation letztendlich bedienen sollen, wurde auf LerntherapeutInnen, Lehrkräfte und Eltern von Betroffenen Kindern festgelegt. Diese bringen sehr unterschiedliche Kenntnisse im Umgang mit Software bzw. Webanwendungen mit. Haben die NutzerInnen Schwierigkeiten, das Programm zu bedienen, führt das schnell zu Frust und letztendlich dazu, dass die Software am Ende überhaupt nicht genutzt, und die Aufgabe weiterhin manuell erledigt wird. \tocite{user experience wichtig} Das Ziel war also die Oberfläche so einfach und intuitiv wie möglich zu gestalten, um Allen NutzerInnen einen reibungslosen Umgang mit der Applikation zu ermöglichen. Es wurde daher beschlossen eine Webbrowser-basierte Anwendung zu entwickeln, die plattformunabhängig funktioniert. Damit wurden von Anfang an Hürden, wie die Bindung an ein bestimmtes Betriebssystem und die Notwendigkeit der Installation von Software beseitigt. Die Applikation kann von jedem Computer mit Internetverbindung und Webbrowser sofort benutzt werden. Durch die Verwendung von Nutzerkonten sind nutzerspezifische Einstellungen überall gleichermaßen abrufbar.\\

Im Evaluationsteil am Ende der Arbeit wurde die Benutzbarkeit des Gesamtsystems anhand eines Nutzertests bewertet. Hierfür wurden als ProbandInnen sowohl Experten (Lerntherapeutinnen) als auch Laien herangezogen. Für die Gruppe der Laien kam praktisch Jeder infrage, da in der Zielgruppe Elternteile enthalten sind und diese aus den verschiedensten Berufsfeldern sowie unterschiedlichen Altersgruppen stammen können. Zunächst wurde ein Pilottest ausgeführt, um Mängel im Design des Tests, sowie Fehler der Webanwendung, die nicht im Zusammenhang mit der Usability standen im Vorfeld zu finden und zu beheben. Im Anschluss wurde der Nutzertest mit sieben ProbandInnen ausgeführt und die Ergebnisse zusammengetragen und bewertet.\\

In den Folgenden Kapiteln werden zunächst die Grundlagen zur Lese-Rechtschreibschwäche sowie zu linguistischen Begriffen und Technologien erklärt. Danach wird der Prozess der Planung und Entwicklung der Software beschrieben. Detaillierte Beschreibungen zu einzelnen Komponenten und zum Zusammenspiel dieser sollen verdeutlichen, wie das System funktioniert. Den Abschluss der Arbeit bildet die Evaluation anhand von Beispielen mit der Applikation produzierter Texte und der Beschreibung und Durchführung des Nutzertests. Hier wird gezeigt, wie der entwickelte Prototyp im tatsächlichen Einsatz zu bewerten ist. Die Evaluation zeigt sowohl Stärken als auch Schwächen des Systems auf, die in einem Weiteren Einsatz noch zu beheben sind. Die Flexibilität des Gesamtsystems schaffte zudem viele anregende neue Ideen, die im Anschluss an diese Arbeit gut realisierbar sind.