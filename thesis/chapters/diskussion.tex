% !TEX root = ../ausarbeitung.tex

\chapter{Diskussion}
Die Diskussion kann als Teil des Evaluations- oder Schlusskapitels oder als eigenständiges Kapitel aufgeführt werden. Wichtig ist, dass Sie Ihre Evaluationsergebnisse realistisch einschätzen und ins Verhältnis zum Stand der Technik setzen. Achten Sie besonders darauf, aus den Daten Ihrer Evaluation keine Wunschergebnisse abzulesen, die nicht in den Daten sind (wenn Ihre Testnutzer Ihr neuimplementiertes System nicht besser bedienen können als ein vorhandenes, dann ist das eben so). Gerade unerwartete bzw. \qq{negative} Ergebnisse (z.B. das neue System ist nicht besser als vorhandene) bringen wissenschaftliche Erkenntnisse: man stellt damit fest, dass der gewählte Weg nicht zum gewünschten Ergebnis führt und man generiert damit neue Fragen, z.B. warum der Weg nicht funktioniert hat, obwohl er vor dem Test als überlegen erachtet wurde.

Es kann auch sein, dass verschiedene Evaluationsformen Unterschiede offenbaren. Z.B. kann es sein, dass die Nutzbarkeit des implementierten Systems nicht besser ist als bei anderen Ansätzen, aber dass es deutlich einfacher zu warten ist.


Im letzten Teil runden Sie Ihre Arbeit ab, in dem Sie Ihre Argumentation aus der Einleitung aufgreifen und mit konkreten Daten aus Ihrem Hauptteil und der Evaluation untermauern. Auch hier können Sie Bezug zur Literatur nehmen. Am Ende sollten Sie einen Ausblick über weitere Forschungsthemen geben. Dabei aufpassen, dass es nicht so klingt wie \qq{mir ist die Zeit ausgegangen und folgendes habe ich nicht mehr geschafft}. Eine gute wissenschaftliche Arbeit wirft mehr Fragen auf als sie beantwortet. Es sollte also eher klingen nach \qq{meine Arbeit hat \dots gezeigt. Daraus ergeben sich weitere interessante Fragen \dots}.

\section{Ausblick}

datenbank: PHONOLEX als Grundlage (statt oder zusaetzlich zu CELEX)
